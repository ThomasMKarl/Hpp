\documentclass[12pt]{scrartcl}
\usepackage[ngerman]{babel}
\usepackage[T1]{fontenc}
\usepackage[utf8]{inputenc}
\usepackage{lmodern}
\usepackage{graphicx}
\usepackage{setspace}
\usepackage{fancyhdr}
\usepackage{lastpage}
\usepackage{amsmath}
\usepackage{pgf}
\usepackage{caption}
\usepackage{import}

\pagestyle{fancy}
\fancyhead[L]{\large Paralleles Ising Modell}
\fancyhead[R]{Thomas Karl}
\fancyfoot[C]{Seite \arabic{page} von \pageref{LastPage}}
\renewcommand{\footrulewidth}{0.4pt}

%\usepackage[colorlinks=true,
%            linkcolor=black,
%            urlcolor=blue,
%            citecolor=gray]{hyperref}
            
\setlength\parindent{0pt}

\begin{document}

Im Ising-Modell wird angenommen, dass die Spins, welche das magnetische Moment der Atome oder Ionen bestimmen, nur zwei diskrete Zustände annehmen können (Spinwert $\pm 1$). Die Richtung im Raum bleibt aber offen.

Wir minimieren mittels Metropolis-Algorithmus den Hamiltonian:
\begin{equation}
H = -\frac{1}{2}\sum_{ij} J_{ij}s^z_is^z_j - B\sum_{\text{i=1}}^Vs_i^{z} ~~~ \text{mit} ~ s_i^z = \pm 1
\end{equation}

In diesem Fall ist die Kopplungskonstante $J_{ij}=J=1$ und das Magnetfeld ist $B = 0$. 

Der Anfangszustand ist zufaellig gew\"hlt: In zwei Dimensionen mit periodischen Randbedingungen sind insgesamt $V = N_x\cdot N_y$ (Volumen) Spins entweder $1$ oder $-1$. 

Beim Metropolis-Algorithmus werden zuf\"allig Spins ge\"ndert um neue Konfigurationen zu erzeugen. Eine neue Konfiguration wird mit der Wahrscheinlichkeit 

\begin{equation}
p = \min\left( 1, \exp\left(-\frac{\Delta E}{k_b\cdot T}\right) \right)
\end{equation}

angenommen. $\Delta E$ ist dabei die Differenz der Energie zum vorherigen Zustand, $k_b$ die Boltzmann-Konstante und $T$ die Temperatur, die auf $T=\frac{1}{10k_b}$ gesetzt wird. Bei dieser Temperatur sollten sich alle Spins gleich ausrichten und die Simulation liefert pro Volumen die Gesamtenergie $-2$.

Es wurden jeweils 10 Simulationen mit 10000 Schritten durchgef\"uhrt und die Laufzeit gemittelt. Diese wurden doppelt-logarithmisch f\"ur verschiedene Kernzahlen gegen die L\"ange des Gitters $s=N_x=N_y=\sqrt{V}$ aufgetragen (\textit{siehe Graph}).

Der Algorithmus funktioniert nur, falls es sich bei der Anzahl der Kerne um eine Quadratzahl handelt.

Da mit $s$ das Volumen quadratisch steigt, also auch die Zahl der zu bearbeitenden Spins, hat der Algorithmus das Laufzeitverhalten $\mathcal{O}(s^2)$.

Die Fitkonstante im Exponenten liefert nicht wie erwartet $2$. Das liegt daran, dass beim Fitten davon ausgegangen wurde, dass Anteile, die linear mit $s$ wachsen, vernachl\"ssigbar seien. Gerade f\"ur kleine Gittergr\"o\ss en trifft dies aber nicht zu. 

Setzt man den Exponenten auf einen festen Wert und fittet nur den Vorfaktor, so ergeben sich Speedups von 3,5 und 4,84. Die Abweichung zum maximalen Speedup 6 im zweiten Fall ergibt sich dadurch, dass die Simulation mit einer Quadratzahl von Prozessen (also 9) gestartet werden muss. Bei der H\"alfte der Zeit bleiben also die H\"alfte der Kerne unbesch\"aftigt ($6 - 1,5 = 4,5$).

\begin{figure}
\scalebox{1.1}{\input{ising.pgf}}
\caption*{Verbrauchte Rechenzeit gegen die L\"ange des Gitters f\"ur alle drei Kernzahlen und Fitfunktionen auf doppelt-logarithmischer Skala.}
\end{figure}

\end{document}