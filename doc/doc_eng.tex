\documentclass[headsepline=3pt,headinclude=true,14pt]{scrartcl}
\usepackage{scrlayer-scrpage}
\usepackage{mwe}

\usepackage[origlayout=true,automark,colors={rz}]{URpagestyles}

\DeclareMathSizes{18}{18}{18}{18}
\usepackage{float}
\usepackage{upgreek}
\usepackage{color}
\usepackage{paralist}
\usepackage[left=2cm,right=2cm,top=1cm,bottom=3cm,includeheadfoot]{geometry}
\usepackage{listings}
\usepackage{setspace} 
\usepackage{graphicx}
\usepackage{polyglossia}
\setmainlanguage{english}
\usepackage{booktabs}
\usepackage{fontspec}
%\setmainfont{Frutiger Next LT W1G}
\usepackage{amsmath}
\usepackage{amsfonts}
\usepackage{amssymb}
\usepackage{xcolor}
%\usepackage{fancyhdr}
\usepackage{lastpage}
\usepackage{caption}
\usepackage{tocstyle}
\newtocstyle[KOMAlike][leaders]{alldotted}{}
\usetocstyle{alldotted}
\usepackage[absolute]{textpos}

\usepackage{lipsum}

\usepackage[backend=biber,style=authoryear-ibid]{biblatex}
%\addbibresource{../lit/lit.bib}

\parskip=1\baselineskip
\parindent = 0pt

\usepackage[colorlinks=true,
            linkcolor=black,
            urlcolor=blue,
            citecolor=red]{hyperref}
\usepackage{array}            
            
\usepackage{makeidx}
\makeindex

\renewcommand*{\familydefault}{\sfdefault}
\newcommand*{\pck}[1]{\texttt{#1}}
\newcommand*{\code}[1]{\texttt{#1}}
\newcommand*{\repl}[1]{\textrm{\textit{#1}}}
\newcommand{\cmd}[1]{\par\medskip\noindent\fbox{\ttfamily#1}\par\medskip\noindent}
\setcounter{secnumdepth}{\sectionnumdepth}
\newsavebox{\remarkbox}
\sbox{\remarkbox}{\emph{Anmerkung:~}}
\newcounter{iterator}
\usepackage{colortbl}
%-------------------------------------------------------------------------------------------------------------

\cfoot*{\thepage\ of \pageref{LastPage}}% the pagenumber in the center of the foot, also on plain pages
\ihead*{ }% Name and title beneath each other in the inner part of the foot
\ohead*{\headmark}
\automark[subsection]{section}

\title{Simulation of the 2d Ising model using \textit{Metropolis Monte-Carlo algorithm} in \textit{C++/MPI}}
\subtitle{Documentation}
\date{\today}
\author{\textbf{Thomas Karl}%\\Matriculation Number: 1631195\\ \textbf{Supervisor: Prof. Gunnar S. Bali}
}

\renewcommand{\titlepagestyle}{URtitle}       

\lstset{%,
  backgroundcolor=\color{white},   % choose the background color; you must add \usepackage{color} or \usepackage{xcolor}
  basicstyle=\footnotesize,        % the size of the fonts that are used for the code
  breakatwhitespace=false,         % sets if automatic breaks should only happen at whitespace
  breaklines=true,                 % sets automatic line breaking
  captionpos=b,                    % sets the caption-position to bottom
  commentstyle=\color{green},    % comment style
  deletekeywords={...},            % if you want to delete keywords from the given language
  escapeinside={\%*}{*)},          % if you want to add LaTeX within your code
  extendedchars=true,              % lets you use non-ASCII characters; for 8-bits encodings only, does not work with UTF-8
  frame=single,	                   % adds a frame around the code
  keepspaces=true,                 % keeps spaces in text, useful for keeping indentation of code (possibly needs columns=flexible)
  keywordstyle=\color{blue},       % keyword style
  language=Octave,                 % the language of the code
  otherkeywords={*,...},           % if you want to add more keywords to the set
  numbers=left,                    % where to put the line-numbers; possible values are (none, left, right)
  numbersep=5pt,                   % how far the line-numbers are from the code
  numberstyle=\tiny\color{gray}, % the style that is used for the line-numbers
  rulecolor=\color{black},         % if not set, the frame-color may be changed on line-breaks within not-black text (e.g. comments (green here))
  showspaces=false,                % show spaces everywhere adding particular underscores; it overrides 'showstringspaces'
  showstringspaces=false,          % underline spaces within strings only
  showtabs=false,                  % show tabs within strings adding particular underscores
  stepnumber=2,                    % the step between two line-numbers. If it's 1, each line will be numbered
  tabsize=2,	                   % sets default tabsize to 2 spaces
  title=\lstname                   % show the filename of files included with \lstinputlisting; also try caption instead of title
}


\begin{document}
\begin{onehalfspace}
\maketitle

\section{Setup}
Im Ising-Modell wird angenommen, dass die Spins, welche das magnetische Moment der Atome oder Ionen bestimmen, nur zwei diskrete Zustände annehmen können (Spinwert $\pm 1$). Die Richtung im Raum bleibt aber offen.

Using the Metropolis-Algorithmus we minimize the hamiltonian,
\begin{equation}
H = -\frac{1}{2}\sum_{ij} J_{ij}s^z_is^z_j - B\sum_{\text{i=1}}^Vs_i^{z} ~~~ \text{with} ~ s_i^z = \pm 1.
\end{equation}

For simplification we set the coupling constant $J_{ij}=J=1$ und the magnetic field $B = 0$. 

The initial state is random: In two dimensions with periodic boundary conditions the $V = N_x\cdot N_y$ (volume) spins are $1$ or $-1$. 

\newpage
The Metropolis-algorithm flips spins randomly to create new configurations. A new configuration gets chosen with the probability, 

\begin{equation}
p = \min\left( 1, \exp\left(-\frac{\Delta E}{k_b\cdot T}\right) \right).
\end{equation}

$\Delta E$ denotes the difference between the energy of the new and the last configuration, $k_b$ is the Boltzmann constant and $T$ is the temperature, which we set to $T=\frac{1}{10k_b}$. At this temperature all spins should align equally and the simulation computes the minimum energy as $-2J$.

We require a quadratic number of CPU cores and a number of spins so, that each process has the same number in its memory  (Abb. \ref{ranks}). All spins are stored in a linear array. Each core stores its part of the grid including the bounds. Each core fulfills an update process independently and computes its part of the overall energy $E$. So  we have four seperated configurations. After that, one core has to make a sum reduction to add all partial energies and broadcast the sum over all cores.

We hav to make an exchange of the bonds before the next Monte-Carlo step. The inner bond (blue in the picture) has to be written in the outer bond of the neighbouring (periodic boundary conditions) grid (red in the picture). Rank 1 for example has to write its right rim in the left one of its right neighbour (Rank 0). Therefore, each prcess is part of a vertical and ahorizontal ring. The exchange of bonds lies one order of magnitude below the complexity of the updates. Therefore, the exchange should be negliable for a sufficiently large grid.

In order to avoid a \textit{deadlock}, the diagonal processes are sending at first, the other ones are receiving. After that they switch.

\begin{figure}
% Graphic for TeX using PGF
% Title: /home/thomas/Diagram1.dia
% Creator: Dia v0.97+git
% CreationDate: Thu Dec 20 17:03:26 2018
% For: thomas
% \usepackage{tikz}
% The following commands are not supported in PSTricks at present
% We define them conditionally, so when they are implemented,
% this pgf file will use them.
\ifx\du\undefined
  \newlength{\du}
\fi
\setlength{\du}{15\unitlength}
\begin{Large}
\begin{tikzpicture}[even odd rule]
\textbf{
\pgftransformxscale{1.000000}
\pgftransformyscale{-1.000000}
\definecolor{dialinecolor}{rgb}{0.000000, 0.000000, 0.000000}
\pgfsetstrokecolor{dialinecolor}
\pgfsetstrokeopacity{1.000000}
\definecolor{diafillcolor}{rgb}{1.000000, 1.000000, 1.000000}
\pgfsetfillcolor{diafillcolor}
\pgfsetfillopacity{1.000000}
\pgfsetlinewidth{0.100000\du}
\pgfsetdash{}{0pt}
\pgfsetmiterjoin
{\pgfsetcornersarced{\pgfpoint{0.000000\du}{0.000000\du}}\definecolor{diafillcolor}{rgb}{1.000000, 1.000000, 1.000000}
\pgfsetfillcolor{diafillcolor}
\pgfsetfillopacity{1.000000}
\fill (15.350000\du,10.150000\du)--(15.350000\du,24.150000\du)--(30.650000\du,24.150000\du)--(30.650000\du,10.150000\du)--cycle;
}{\pgfsetcornersarced{\pgfpoint{0.000000\du}{0.000000\du}}\definecolor{dialinecolor}{rgb}{0.000000, 0.000000, 0.000000}
\pgfsetstrokecolor{dialinecolor}
\pgfsetstrokeopacity{1.000000}
\draw (15.350000\du,10.150000\du)--(15.350000\du,24.150000\du)--(30.650000\du,24.150000\du)--(30.650000\du,10.150000\du)--cycle;
}% setfont left to latex
\definecolor{dialinecolor}{rgb}{0.000000, 0.000000, 0.000000}
\pgfsetstrokecolor{dialinecolor}
\pgfsetstrokeopacity{1.000000}
\definecolor{diafillcolor}{rgb}{0.000000, 0.000000, 0.000000}
\pgfsetfillcolor{diafillcolor}
\pgfsetfillopacity{1.000000}
\node[anchor=base,inner sep=0pt, outer sep=0pt,color=dialinecolor] at (23.000000\du,17.345000\du){};
\pgfsetlinewidth{0.100000\du}
\pgfsetdash{}{0pt}
\pgfsetmiterjoin
{\pgfsetcornersarced{\pgfpoint{0.000000\du}{0.000000\du}}\definecolor{diafillcolor}{rgb}{1.000000, 1.000000, 1.000000}
\pgfsetfillcolor{diafillcolor}
\pgfsetfillopacity{1.000000}
\fill (30.700000\du,10.150000\du)--(30.700000\du,24.150000\du)--(46.000000\du,24.150000\du)--(46.000000\du,10.150000\du)--cycle;
}{\pgfsetcornersarced{\pgfpoint{0.000000\du}{0.000000\du}}\definecolor{dialinecolor}{rgb}{0.000000, 0.000000, 0.000000}
\pgfsetstrokecolor{dialinecolor}
\pgfsetstrokeopacity{1.000000}
\draw (30.700000\du,10.150000\du)--(30.700000\du,24.150000\du)--(46.000000\du,24.150000\du)--(46.000000\du,10.150000\du)--cycle;
}% setfont left to latex
\definecolor{dialinecolor}{rgb}{0.000000, 0.000000, 0.000000}
\pgfsetstrokecolor{dialinecolor}
\pgfsetstrokeopacity{1.000000}
\definecolor{diafillcolor}{rgb}{0.000000, 0.000000, 0.000000}
\pgfsetfillcolor{diafillcolor}
\pgfsetfillopacity{1.000000}
\node[anchor=base,inner sep=0pt, outer sep=0pt,color=dialinecolor] at (38.350000\du,17.345000\du){};
\pgfsetlinewidth{0.100000\du}
\pgfsetdash{}{0pt}
\pgfsetmiterjoin
{\pgfsetcornersarced{\pgfpoint{0.000000\du}{0.000000\du}}\definecolor{diafillcolor}{rgb}{1.000000, 1.000000, 1.000000}
\pgfsetfillcolor{diafillcolor}
\pgfsetfillopacity{1.000000}
\fill (15.400000\du,24.150000\du)--(15.400000\du,40.000000\du)--(30.650000\du,40.000000\du)--(30.650000\du,24.150000\du)--cycle;
}{\pgfsetcornersarced{\pgfpoint{0.000000\du}{0.000000\du}}\definecolor{dialinecolor}{rgb}{0.000000, 0.000000, 0.000000}
\pgfsetstrokecolor{dialinecolor}
\pgfsetstrokeopacity{1.000000}
\draw (15.400000\du,24.150000\du)--(15.400000\du,40.000000\du)--(30.650000\du,40.000000\du)--(30.650000\du,24.150000\du)--cycle;
}% setfont left to latex
\definecolor{dialinecolor}{rgb}{0.000000, 0.000000, 0.000000}
\pgfsetstrokecolor{dialinecolor}
\pgfsetstrokeopacity{1.000000}
\definecolor{diafillcolor}{rgb}{0.000000, 0.000000, 0.000000}
\pgfsetfillcolor{diafillcolor}
\pgfsetfillopacity{1.000000}
\node[anchor=base,inner sep=0pt, outer sep=0pt,color=dialinecolor] at (23.025000\du,32.270000\du){};
\pgfsetlinewidth{0.100000\du}
\pgfsetdash{}{0pt}
\pgfsetmiterjoin
{\pgfsetcornersarced{\pgfpoint{0.000000\du}{0.000000\du}}\definecolor{diafillcolor}{rgb}{1.000000, 1.000000, 1.000000}
\pgfsetfillcolor{diafillcolor}
\pgfsetfillopacity{1.000000}
\fill (30.700000\du,24.100000\du)--(30.700000\du,40.000000\du)--(46.000000\du,40.000000\du)--(46.000000\du,24.100000\du)--cycle;
}{\pgfsetcornersarced{\pgfpoint{0.000000\du}{0.000000\du}}\definecolor{dialinecolor}{rgb}{0.000000, 0.000000, 0.000000}
\pgfsetstrokecolor{dialinecolor}
\pgfsetstrokeopacity{1.000000}
\draw (30.700000\du,24.100000\du)--(30.700000\du,40.000000\du)--(46.000000\du,40.000000\du)--(46.000000\du,24.100000\du)--cycle;
}% setfont left to latex
\definecolor{dialinecolor}{rgb}{0.000000, 0.000000, 0.000000}
\pgfsetstrokecolor{dialinecolor}
\pgfsetstrokeopacity{1.000000}
\definecolor{diafillcolor}{rgb}{0.000000, 0.000000, 0.000000}
\pgfsetfillcolor{diafillcolor}
\pgfsetfillopacity{1.000000}
\node[anchor=base,inner sep=0pt, outer sep=0pt,color=dialinecolor] at (38.350000\du,32.245000\du){};
\pgfsetlinewidth{0.100000\du}
\pgfsetdash{}{0pt}
\pgfsetbuttcap
\pgfsetmiterjoin
\pgfsetlinewidth{0.100000\du}
\pgfsetbuttcap
\pgfsetmiterjoin
\pgfsetdash{}{0pt}
\definecolor{diafillcolor}{rgb}{1.000000, 1.000000, 1.000000}
\pgfsetfillcolor{diafillcolor}
\pgfsetfillopacity{1.000000}
\definecolor{dialinecolor}{rgb}{1.000000, 0.000000, 0.000000}
\pgfsetstrokecolor{dialinecolor}
\pgfsetstrokeopacity{1.000000}
\pgfpathmoveto{\pgfpoint{31.083333\du}{10.200000\du}}
\pgfpathlineto{\pgfpoint{32.416667\du}{10.200000\du}}
\pgfpathcurveto{\pgfpoint{32.600762\du}{10.200000\du}}{\pgfpoint{32.750000\du}{13.300426\du}}{\pgfpoint{32.750000\du}{17.125000\du}}
\pgfpathcurveto{\pgfpoint{32.750000\du}{20.949574\du}}{\pgfpoint{32.600762\du}{24.050000\du}}{\pgfpoint{32.416667\du}{24.050000\du}}
\pgfpathlineto{\pgfpoint{31.083333\du}{24.050000\du}}
\pgfpathcurveto{\pgfpoint{30.899238\du}{24.050000\du}}{\pgfpoint{30.750000\du}{20.949574\du}}{\pgfpoint{30.750000\du}{17.125000\du}}
\pgfpathcurveto{\pgfpoint{30.750000\du}{13.300426\du}}{\pgfpoint{30.899238\du}{10.200000\du}}{\pgfpoint{31.083333\du}{10.200000\du}}
\pgfpathclose
\pgfusepath{fill,stroke}
% setfont left to latex
\definecolor{dialinecolor}{rgb}{0.000000, 0.000000, 0.000000}
\pgfsetstrokecolor{dialinecolor}
\pgfsetstrokeopacity{1.000000}
\definecolor{diafillcolor}{rgb}{0.000000, 0.000000, 0.000000}
\pgfsetfillcolor{diafillcolor}
\pgfsetfillopacity{1.000000}
\node[anchor=base,inner sep=0pt, outer sep=0pt,color=dialinecolor] at (31.750000\du,17.325000\du){};
\pgfsetlinewidth{0.100000\du}
\pgfsetdash{}{0pt}
\pgfsetbuttcap
\pgfsetmiterjoin
\pgfsetlinewidth{0.100000\du}
\pgfsetbuttcap
\pgfsetmiterjoin
\pgfsetdash{}{0pt}
\definecolor{diafillcolor}{rgb}{1.000000, 1.000000, 1.000000}
\pgfsetfillcolor{diafillcolor}
\pgfsetfillopacity{1.000000}
\definecolor{dialinecolor}{rgb}{1.000000, 0.000000, 0.000000}
\pgfsetstrokecolor{dialinecolor}
\pgfsetstrokeopacity{1.000000}
\pgfpathmoveto{\pgfpoint{17.950000\du}{24.250000\du}}
\pgfpathlineto{\pgfpoint{28.150000\du}{24.250000\du}}
\pgfpathcurveto{\pgfpoint{29.558327\du}{24.250000\du}}{\pgfpoint{30.700000\du}{24.664136\du}}{\pgfpoint{30.700000\du}{25.175000\du}}
\pgfpathcurveto{\pgfpoint{30.700000\du}{25.685864\du}}{\pgfpoint{29.558327\du}{26.100000\du}}{\pgfpoint{28.150000\du}{26.100000\du}}
\pgfpathlineto{\pgfpoint{17.950000\du}{26.100000\du}}
\pgfpathcurveto{\pgfpoint{16.541673\du}{26.100000\du}}{\pgfpoint{15.400000\du}{25.685864\du}}{\pgfpoint{15.400000\du}{25.175000\du}}
\pgfpathcurveto{\pgfpoint{15.400000\du}{24.664136\du}}{\pgfpoint{16.541673\du}{24.250000\du}}{\pgfpoint{17.950000\du}{24.250000\du}}
\pgfpathclose
\pgfusepath{fill,stroke}
% setfont left to latex
\definecolor{dialinecolor}{rgb}{0.000000, 0.000000, 0.000000}
\pgfsetstrokecolor{dialinecolor}
\pgfsetstrokeopacity{1.000000}
\definecolor{diafillcolor}{rgb}{0.000000, 0.000000, 0.000000}
\pgfsetfillcolor{diafillcolor}
\pgfsetfillopacity{1.000000}
\node[anchor=base,inner sep=0pt, outer sep=0pt,color=dialinecolor] at (23.050000\du,25.375000\du){};
\pgfsetlinewidth{0.100000\du}
\pgfsetdash{}{0pt}
\pgfsetbuttcap
\pgfsetmiterjoin
\pgfsetlinewidth{0.100000\du}
\pgfsetbuttcap
\pgfsetmiterjoin
\pgfsetdash{}{0pt}
\definecolor{diafillcolor}{rgb}{1.000000, 1.000000, 1.000000}
\pgfsetfillcolor{diafillcolor}
\pgfsetfillopacity{1.000000}
\definecolor{dialinecolor}{rgb}{1.000000, 0.000000, 0.000000}
\pgfsetstrokecolor{dialinecolor}
\pgfsetstrokeopacity{1.000000}
\pgfpathmoveto{\pgfpoint{13.633333\du}{10.120000\du}}
\pgfpathlineto{\pgfpoint{14.966667\du}{10.120000\du}}
\pgfpathcurveto{\pgfpoint{15.150762\du}{10.120000\du}}{\pgfpoint{15.300000\du}{13.227142\du}}{\pgfpoint{15.300000\du}{17.060000\du}}
\pgfpathcurveto{\pgfpoint{15.300000\du}{20.892858\du}}{\pgfpoint{15.150762\du}{24.000000\du}}{\pgfpoint{14.966667\du}{24.000000\du}}
\pgfpathlineto{\pgfpoint{13.633333\du}{24.000000\du}}
\pgfpathcurveto{\pgfpoint{13.449238\du}{24.000000\du}}{\pgfpoint{13.300000\du}{20.892858\du}}{\pgfpoint{13.300000\du}{17.060000\du}}
\pgfpathcurveto{\pgfpoint{13.300000\du}{13.227142\du}}{\pgfpoint{13.449238\du}{10.120000\du}}{\pgfpoint{13.633333\du}{10.120000\du}}
\pgfpathclose
\pgfusepath{fill,stroke}
% setfont left to latex
\definecolor{dialinecolor}{rgb}{0.000000, 0.000000, 0.000000}
\pgfsetstrokecolor{dialinecolor}
\pgfsetstrokeopacity{1.000000}
\definecolor{diafillcolor}{rgb}{0.000000, 0.000000, 0.000000}
\pgfsetfillcolor{diafillcolor}
\pgfsetfillopacity{1.000000}
\node[anchor=base,inner sep=0pt, outer sep=0pt,color=dialinecolor] at (14.300000\du,17.260000\du){};
\pgfsetlinewidth{0.100000\du}
\pgfsetdash{}{0pt}
\pgfsetbuttcap
\pgfsetmiterjoin
\pgfsetlinewidth{0.100000\du}
\pgfsetbuttcap
\pgfsetmiterjoin
\pgfsetdash{}{0pt}
\definecolor{diafillcolor}{rgb}{1.000000, 1.000000, 1.000000}
\pgfsetfillcolor{diafillcolor}
\pgfsetfillopacity{1.000000}
\definecolor{dialinecolor}{rgb}{1.000000, 0.000000, 0.000000}
\pgfsetstrokecolor{dialinecolor}
\pgfsetstrokeopacity{1.000000}
\pgfpathmoveto{\pgfpoint{17.955000\du}{8.220000\du}}
\pgfpathlineto{\pgfpoint{28.155000\du}{8.220000\du}}
\pgfpathcurveto{\pgfpoint{29.563327\du}{8.220000\du}}{\pgfpoint{30.705000\du}{8.634136\du}}{\pgfpoint{30.705000\du}{9.145000\du}}
\pgfpathcurveto{\pgfpoint{30.705000\du}{9.655864\du}}{\pgfpoint{29.563327\du}{10.070000\du}}{\pgfpoint{28.155000\du}{10.070000\du}}
\pgfpathlineto{\pgfpoint{17.955000\du}{10.070000\du}}
\pgfpathcurveto{\pgfpoint{16.546673\du}{10.070000\du}}{\pgfpoint{15.405000\du}{9.655864\du}}{\pgfpoint{15.405000\du}{9.145000\du}}
\pgfpathcurveto{\pgfpoint{15.405000\du}{8.634136\du}}{\pgfpoint{16.546673\du}{8.220000\du}}{\pgfpoint{17.955000\du}{8.220000\du}}
\pgfpathclose
\pgfusepath{fill,stroke}
% setfont left to latex
\definecolor{dialinecolor}{rgb}{0.000000, 0.000000, 0.000000}
\pgfsetstrokecolor{dialinecolor}
\pgfsetstrokeopacity{1.000000}
\definecolor{diafillcolor}{rgb}{0.000000, 0.000000, 0.000000}
\pgfsetfillcolor{diafillcolor}
\pgfsetfillopacity{1.000000}
\node[anchor=base,inner sep=0pt, outer sep=0pt,color=dialinecolor] at (23.055000\du,9.345000\du){};
\pgfsetlinewidth{0.100000\du}
\pgfsetdash{}{0pt}
\pgfsetbuttcap
{
\definecolor{diafillcolor}{rgb}{0.000000, 0.000000, 1.000000}
\pgfsetfillcolor{diafillcolor}
\pgfsetfillopacity{1.000000}
% was here!!!
\pgfsetarrowsend{stealth}
\definecolor{dialinecolor}{rgb}{0.000000, 0.000000, 1.000000}
\pgfsetstrokecolor{dialinecolor}
\pgfsetstrokeopacity{1.000000}
\pgfpathmoveto{\pgfpoint{14.797626\du}{18.198967\du}}
\pgfpatharc{117}{69}{35.992542\du and 35.992542\du}
\pgfusepath{stroke}
}
\pgfsetlinewidth{0.100000\du}
\pgfsetdash{}{0pt}
\pgfsetbuttcap
\pgfsetmiterjoin
\pgfsetlinewidth{0.100000\du}
\pgfsetbuttcap
\pgfsetmiterjoin
\pgfsetdash{}{0pt}
\definecolor{diafillcolor}{rgb}{1.000000, 1.000000, 1.000000}
\pgfsetfillcolor{diafillcolor}
\pgfsetfillopacity{1.000000}
\definecolor{dialinecolor}{rgb}{0.000000, 0.000000, 1.000000}
\pgfsetstrokecolor{dialinecolor}
\pgfsetstrokeopacity{1.000000}
\pgfpathmoveto{\pgfpoint{44.288333\du}{10.220000\du}}
\pgfpathlineto{\pgfpoint{45.621667\du}{10.220000\du}}
\pgfpathcurveto{\pgfpoint{45.805762\du}{10.220000\du}}{\pgfpoint{45.955000\du}{13.320426\du}}{\pgfpoint{45.955000\du}{17.145000\du}}
\pgfpathcurveto{\pgfpoint{45.955000\du}{20.969574\du}}{\pgfpoint{45.805762\du}{24.070000\du}}{\pgfpoint{45.621667\du}{24.070000\du}}
\pgfpathlineto{\pgfpoint{44.288333\du}{24.070000\du}}
\pgfpathcurveto{\pgfpoint{44.104238\du}{24.070000\du}}{\pgfpoint{43.955000\du}{20.969574\du}}{\pgfpoint{43.955000\du}{17.145000\du}}
\pgfpathcurveto{\pgfpoint{43.955000\du}{13.320426\du}}{\pgfpoint{44.104238\du}{10.220000\du}}{\pgfpoint{44.288333\du}{10.220000\du}}
\pgfpathclose
\pgfusepath{fill,stroke}
% setfont left to latex
\definecolor{dialinecolor}{rgb}{0.000000, 0.000000, 0.000000}
\pgfsetstrokecolor{dialinecolor}
\pgfsetstrokeopacity{1.000000}
\definecolor{diafillcolor}{rgb}{0.000000, 0.000000, 0.000000}
\pgfsetfillcolor{diafillcolor}
\pgfsetfillopacity{1.000000}
\node[anchor=base,inner sep=0pt, outer sep=0pt,color=dialinecolor] at (44.955000\du,17.345000\du){};
\pgfsetlinewidth{0.100000\du}
\pgfsetdash{}{0pt}
\pgfsetbuttcap
\pgfsetmiterjoin
\pgfsetlinewidth{0.100000\du}
\pgfsetbuttcap
\pgfsetmiterjoin
\pgfsetdash{}{0pt}
\definecolor{diafillcolor}{rgb}{1.000000, 1.000000, 1.000000}
\pgfsetfillcolor{diafillcolor}
\pgfsetfillopacity{1.000000}
\definecolor{dialinecolor}{rgb}{0.000000, 0.000000, 1.000000}
\pgfsetstrokecolor{dialinecolor}
\pgfsetstrokeopacity{1.000000}
\pgfpathmoveto{\pgfpoint{17.855000\du}{37.920000\du}}
\pgfpathlineto{\pgfpoint{28.055000\du}{37.920000\du}}
\pgfpathcurveto{\pgfpoint{29.463327\du}{37.920000\du}}{\pgfpoint{30.605000\du}{38.374431\du}}{\pgfpoint{30.605000\du}{38.935000\du}}
\pgfpathcurveto{\pgfpoint{30.605000\du}{39.495569\du}}{\pgfpoint{29.463327\du}{39.950000\du}}{\pgfpoint{28.055000\du}{39.950000\du}}
\pgfpathlineto{\pgfpoint{17.855000\du}{39.950000\du}}
\pgfpathcurveto{\pgfpoint{16.446673\du}{39.950000\du}}{\pgfpoint{15.305000\du}{39.495569\du}}{\pgfpoint{15.475000\du}{38.935000\du}}
\pgfpathcurveto{\pgfpoint{15.305000\du}{38.374431\du}}{\pgfpoint{16.446673\du}{37.920000\du}}{\pgfpoint{17.855000\du}{37.920000\du}}
\pgfpathclose
\pgfusepath{fill,stroke}
% setfont left to latex
\definecolor{dialinecolor}{rgb}{0.000000, 0.000000, 0.000000}
\pgfsetstrokecolor{dialinecolor}
\pgfsetstrokeopacity{1.000000}
\definecolor{diafillcolor}{rgb}{0.000000, 0.000000, 0.000000}
\pgfsetfillcolor{diafillcolor}
\pgfsetfillopacity{1.000000}
\node[anchor=base,inner sep=0pt, outer sep=0pt,color=dialinecolor] at (22.955000\du,39.135000\du){};
\pgfsetlinewidth{0.100000\du}
\pgfsetdash{}{0pt}
\pgfsetbuttcap
{
\definecolor{diafillcolor}{rgb}{0.000000, 0.000000, 1.000000}
\pgfsetfillcolor{diafillcolor}
\pgfsetfillopacity{1.000000}
% was here!!!
\pgfsetarrowsend{stealth}
\definecolor{dialinecolor}{rgb}{0.000000, 0.000000, 1.000000}
\pgfsetstrokecolor{dialinecolor}
\pgfsetstrokeopacity{1.000000}
\pgfpathmoveto{\pgfpoint{23.650033\du}{39.149760\du}}
\pgfpatharc{367}{352.3}{113.000000\du and 113.000000\du}
\pgfusepath{stroke}
}
\definecolor{diafillcolor}{rgb}{1.000000, 1.000000, 1.000000}
\pgfsetfillcolor{diafillcolor}
\pgfsetfillopacity{1.000000}
%\fill (21.600000\du,16.300000\du)--(25.497500\du,16.300000\du)--(25.497500\du,17.617500\du)--(21.600000\du,17.617500\du)--cycle;
% setfont left to latex
\definecolor{dialinecolor}{rgb}{0.145098, 0.729412, 0.145098}
\pgfsetstrokecolor{dialinecolor}
\pgfsetstrokeopacity{1.000000}
\definecolor{diafillcolor}{rgb}{0.145098, 0.729412, 0.145098}
\pgfsetfillcolor{diafillcolor}
\pgfsetfillopacity{1.000000}
\node[anchor=base west,inner sep=0pt,outer sep=0pt,color=dialinecolor] at (21.100000\du,17.350000\du){Rank 0};
\definecolor{diafillcolor}{rgb}{1.000000, 1.000000, 1.000000}
\pgfsetfillcolor{diafillcolor}
\pgfsetfillopacity{1.000000}
\fill (37.250000\du,16.300010\du)--(41.147500\du,16.300010\du)--(41.147500\du,17.617510\du)--(37.250000\du,17.617510\du)--cycle;
% setfont left to latex
\definecolor{dialinecolor}{rgb}{0.145098, 0.729412, 0.145098}
\pgfsetstrokecolor{dialinecolor}
\pgfsetstrokeopacity{1.000000}
\definecolor{diafillcolor}{rgb}{0.145098, 0.729412, 0.145098}
\pgfsetfillcolor{diafillcolor}
\pgfsetfillopacity{1.000000}
\node[anchor=base west,inner sep=0pt,outer sep=0pt,color=dialinecolor] at (35.900000\du,17.350000\du){Rank 1};
\definecolor{diafillcolor}{rgb}{1.000000, 1.000000, 1.000000}
\pgfsetfillcolor{diafillcolor}
\pgfsetfillopacity{1.000000}
%\fill (21.400000\du,31.850010\du)--(25.297500\du,31.850010\du)--(25.297500\du,33.167510\du)--(21.400000\du,33.167510\du)--cycle;
% setfont left to latex
\definecolor{dialinecolor}{rgb}{0.145098, 0.729412, 0.145098}
\pgfsetstrokecolor{dialinecolor}
\pgfsetstrokeopacity{1.000000}
\definecolor{diafillcolor}{rgb}{0.145098, 0.729412, 0.145098}
\pgfsetfillcolor{diafillcolor}
\pgfsetfillopacity{1.000000}
\node[anchor=base west,inner sep=0pt,outer sep=0pt,color=dialinecolor] at (21.100000\du,33.100010\du){Rank 2};
\definecolor{diafillcolor}{rgb}{1.000000, 1.000000, 1.000000}
\pgfsetfillcolor{diafillcolor}
\pgfsetfillopacity{1.000000}
\fill (37.250000\du,32.050010\du)--(41.147500\du,32.050010\du)--(41.147500\du,33.367510\du)--(37.250000\du,33.367510\du)--cycle;
% setfont left to latex
\definecolor{dialinecolor}{rgb}{0.145098, 0.729412, 0.145098}
\pgfsetstrokecolor{dialinecolor}
\pgfsetstrokeopacity{1.000000}
\definecolor{diafillcolor}{rgb}{0.145098, 0.729412, 0.145098}
\pgfsetfillcolor{diafillcolor}
\pgfsetfillopacity{1.000000}
\node[anchor=base west,inner sep=0pt,outer sep=0pt,color=dialinecolor] at (35.900000\du,33.100010\du){Rank 3};
% setfont left to latex
\definecolor{dialinecolor}{rgb}{0.000000, 0.000000, 0.000000}
\pgfsetstrokecolor{dialinecolor}
\pgfsetstrokeopacity{1.000000}
\definecolor{diafillcolor}{rgb}{0.000000, 0.000000, 0.000000}
\pgfsetfillcolor{diafillcolor}
\pgfsetfillopacity{1.000000}
\node[anchor=base west,inner sep=0pt,outer sep=0pt,color=dialinecolor] at (23.000000\du,17.150000\du){};
% setfont left to latex
\definecolor{dialinecolor}{rgb}{0.000000, 0.000000, 0.000000}
\pgfsetstrokecolor{dialinecolor}
\pgfsetstrokeopacity{1.000000}
\definecolor{diafillcolor}{rgb}{0.000000, 0.000000, 0.000000}
\pgfsetfillcolor{diafillcolor}
\pgfsetfillopacity{1.000000}
\node[anchor=base west,inner sep=0pt,outer sep=0pt,color=dialinecolor] at (20.000000\du,9.500010\du){~1};
% setfont left to latex
\definecolor{dialinecolor}{rgb}{0.000000, 0.000000, 0.000000}
\pgfsetstrokecolor{dialinecolor}
\pgfsetstrokeopacity{1.000000}
\definecolor{diafillcolor}{rgb}{0.000000, 0.000000, 0.000000}
\pgfsetfillcolor{diafillcolor}
\pgfsetfillopacity{1.000000}
\node[anchor=base west,inner sep=0pt,outer sep=0pt,color=dialinecolor] at (20.000000\du,11.660843\du){-1};
% setfont left to latex
\definecolor{dialinecolor}{rgb}{0.000000, 0.000000, 0.000000}
\pgfsetstrokecolor{dialinecolor}
\pgfsetstrokeopacity{1.000000}
\definecolor{diafillcolor}{rgb}{0.000000, 0.000000, 0.000000}
\pgfsetfillcolor{diafillcolor}
\pgfsetfillopacity{1.000000}
\node[anchor=base west,inner sep=0pt,outer sep=0pt,color=dialinecolor] at (20.000000\du,13.671677\du){~1};
% setfont left to latex
\definecolor{dialinecolor}{rgb}{0.000000, 0.000000, 0.000000}
\pgfsetstrokecolor{dialinecolor}
\pgfsetstrokeopacity{1.000000}
\definecolor{diafillcolor}{rgb}{0.000000, 0.000000, 0.000000}
\pgfsetfillcolor{diafillcolor}
\pgfsetfillopacity{1.000000}
\node[anchor=base west,inner sep=0pt,outer sep=0pt,color=dialinecolor] at (20.000000\du,15.682510\du){~1};
% setfont left to latex
\definecolor{dialinecolor}{rgb}{0.000000, 0.000000, 0.000000}
\pgfsetstrokecolor{dialinecolor}
\pgfsetstrokeopacity{1.000000}
\definecolor{diafillcolor}{rgb}{0.000000, 0.000000, 0.000000}
\pgfsetfillcolor{diafillcolor}
\pgfsetfillopacity{1.000000}
\node[anchor=base west,inner sep=0pt,outer sep=0pt,color=dialinecolor] at (20.000000\du,17.693343\du){~1};
% setfont left to latex
\definecolor{dialinecolor}{rgb}{0.000000, 0.000000, 0.000000}
\pgfsetstrokecolor{dialinecolor}
\pgfsetstrokeopacity{1.000000}
\definecolor{diafillcolor}{rgb}{0.000000, 0.000000, 0.000000}
\pgfsetfillcolor{diafillcolor}
\pgfsetfillopacity{1.000000}
\node[anchor=base west,inner sep=0pt,outer sep=0pt,color=dialinecolor] at (20.000000\du,19.704176\du){-1};
% setfont left to latex
\definecolor{dialinecolor}{rgb}{0.000000, 0.000000, 0.000000}
\pgfsetstrokecolor{dialinecolor}
\pgfsetstrokeopacity{1.000000}
\definecolor{diafillcolor}{rgb}{0.000000, 0.000000, 0.000000}
\pgfsetfillcolor{diafillcolor}
\pgfsetfillopacity{1.000000}
\node[anchor=base west,inner sep=0pt,outer sep=0pt,color=dialinecolor] at (20.000000\du,21.715010\du){~1};
% setfont left to latex
\definecolor{dialinecolor}{rgb}{0.000000, 0.000000, 0.000000}
\pgfsetstrokecolor{dialinecolor}
\pgfsetstrokeopacity{1.000000}
\definecolor{diafillcolor}{rgb}{0.000000, 0.000000, 0.000000}
\pgfsetfillcolor{diafillcolor}
\pgfsetfillopacity{1.000000}
\node[anchor=base west,inner sep=0pt,outer sep=0pt,color=dialinecolor] at (20.000000\du,23.725843\du){~1};
% setfont left to latex
\definecolor{dialinecolor}{rgb}{0.000000, 0.000000, 0.000000}
\pgfsetstrokecolor{dialinecolor}
\pgfsetstrokeopacity{1.000000}
\definecolor{diafillcolor}{rgb}{0.000000, 0.000000, 0.000000}
\pgfsetfillcolor{diafillcolor}
\pgfsetfillopacity{1.000000}
\node[anchor=base west,inner sep=0pt,outer sep=0pt,color=dialinecolor] at (20.000000\du,25.736676\du){-1};
% setfont left to latex
\definecolor{dialinecolor}{rgb}{0.000000, 0.000000, 0.000000}
\pgfsetstrokecolor{dialinecolor}
\pgfsetstrokeopacity{1.000000}
\definecolor{diafillcolor}{rgb}{0.000000, 0.000000, 0.000000}
\pgfsetfillcolor{diafillcolor}
\pgfsetfillopacity{1.000000}
\node[anchor=base west,inner sep=0pt,outer sep=0pt,color=dialinecolor] at (13.550000\du,11.700010\du){~1};
% setfont left to latex
\definecolor{dialinecolor}{rgb}{0.000000, 0.000000, 0.000000}
\pgfsetstrokecolor{dialinecolor}
\pgfsetstrokeopacity{1.000000}
\definecolor{diafillcolor}{rgb}{0.000000, 0.000000, 0.000000}
\pgfsetfillcolor{diafillcolor}
\pgfsetfillopacity{1.000000}
\node[anchor=base west,inner sep=0pt,outer sep=0pt,color=dialinecolor] at (13.550000\du,13.710843\du){~1};
% setfont left to latex
\definecolor{dialinecolor}{rgb}{0.000000, 0.000000, 0.000000}
\pgfsetstrokecolor{dialinecolor}
\pgfsetstrokeopacity{1.000000}
\definecolor{diafillcolor}{rgb}{0.000000, 0.000000, 0.000000}
\pgfsetfillcolor{diafillcolor}
\pgfsetfillopacity{1.000000}
\node[anchor=base west,inner sep=0pt,outer sep=0pt,color=dialinecolor] at (13.550000\du,15.721677\du){-1};
% setfont left to latex
\definecolor{dialinecolor}{rgb}{0.000000, 0.000000, 0.000000}
\pgfsetstrokecolor{dialinecolor}
\pgfsetstrokeopacity{1.000000}
\definecolor{diafillcolor}{rgb}{0.000000, 0.000000, 0.000000}
\pgfsetfillcolor{diafillcolor}
\pgfsetfillopacity{1.000000}
\node[anchor=base west,inner sep=0pt,outer sep=0pt,color=dialinecolor] at (13.550000\du,17.732510\du){-1};
% setfont left to latex
\definecolor{dialinecolor}{rgb}{0.000000, 0.000000, 0.000000}
\pgfsetstrokecolor{dialinecolor}
\pgfsetstrokeopacity{1.000000}
\definecolor{diafillcolor}{rgb}{0.000000, 0.000000, 0.000000}
\pgfsetfillcolor{diafillcolor}
\pgfsetfillopacity{1.000000}
\node[anchor=base west,inner sep=0pt,outer sep=0pt,color=dialinecolor] at (13.550000\du,19.743343\du){~1};
% setfont left to latex
\definecolor{dialinecolor}{rgb}{0.000000, 0.000000, 0.000000}
\pgfsetstrokecolor{dialinecolor}
\pgfsetstrokeopacity{1.000000}
\definecolor{diafillcolor}{rgb}{0.000000, 0.000000, 0.000000}
\pgfsetfillcolor{diafillcolor}
\pgfsetfillopacity{1.000000}
\node[anchor=base west,inner sep=0pt,outer sep=0pt,color=dialinecolor] at (13.550000\du,21.754176\du){-1};
% setfont left to latex
\definecolor{dialinecolor}{rgb}{0.000000, 0.000000, 0.000000}
\pgfsetstrokecolor{dialinecolor}
\pgfsetstrokeopacity{1.000000}
\definecolor{diafillcolor}{rgb}{0.000000, 0.000000, 0.000000}
\pgfsetfillcolor{diafillcolor}
\pgfsetfillopacity{1.000000}
\node[anchor=base west,inner sep=0pt,outer sep=0pt,color=dialinecolor] at (13.550000\du,23.765010\du){~1};
% setfont left to latex
\definecolor{dialinecolor}{rgb}{0.000000, 0.000000, 0.000000}
\pgfsetstrokecolor{dialinecolor}
\pgfsetstrokeopacity{1.000000}
\definecolor{diafillcolor}{rgb}{0.000000, 0.000000, 0.000000}
\pgfsetfillcolor{diafillcolor}
\pgfsetfillopacity{1.000000}
\node[anchor=base west,inner sep=0pt,outer sep=0pt,color=dialinecolor] at (44.255000\du,11.615010\du){~1};
% setfont left to latex
\definecolor{dialinecolor}{rgb}{0.000000, 0.000000, 0.000000}
\pgfsetstrokecolor{dialinecolor}
\pgfsetstrokeopacity{1.000000}
\definecolor{diafillcolor}{rgb}{0.000000, 0.000000, 0.000000}
\pgfsetfillcolor{diafillcolor}
\pgfsetfillopacity{1.000000}
\node[anchor=base west,inner sep=0pt,outer sep=0pt,color=dialinecolor] at (44.255000\du,13.625843\du){~1};
% setfont left to latex
\definecolor{dialinecolor}{rgb}{0.000000, 0.000000, 0.000000}
\pgfsetstrokecolor{dialinecolor}
\pgfsetstrokeopacity{1.000000}
\definecolor{diafillcolor}{rgb}{0.000000, 0.000000, 0.000000}
\pgfsetfillcolor{diafillcolor}
\pgfsetfillopacity{1.000000}
\node[anchor=base west,inner sep=0pt,outer sep=0pt,color=dialinecolor] at (44.255000\du,15.636677\du){-1};
% setfont left to latex
\definecolor{dialinecolor}{rgb}{0.000000, 0.000000, 0.000000}
\pgfsetstrokecolor{dialinecolor}
\pgfsetstrokeopacity{1.000000}
\definecolor{diafillcolor}{rgb}{0.000000, 0.000000, 0.000000}
\pgfsetfillcolor{diafillcolor}
\pgfsetfillopacity{1.000000}
\node[anchor=base west,inner sep=0pt,outer sep=0pt,color=dialinecolor] at (44.255000\du,17.647510\du){-1};
% setfont left to latex
\definecolor{dialinecolor}{rgb}{0.000000, 0.000000, 0.000000}
\pgfsetstrokecolor{dialinecolor}
\pgfsetstrokeopacity{1.000000}
\definecolor{diafillcolor}{rgb}{0.000000, 0.000000, 0.000000}
\pgfsetfillcolor{diafillcolor}
\pgfsetfillopacity{1.000000}
\node[anchor=base west,inner sep=0pt,outer sep=0pt,color=dialinecolor] at (44.255000\du,19.658343\du){~1};
% setfont left to latex
\definecolor{dialinecolor}{rgb}{0.000000, 0.000000, 0.000000}
\pgfsetstrokecolor{dialinecolor}
\pgfsetstrokeopacity{1.000000}
\definecolor{diafillcolor}{rgb}{0.000000, 0.000000, 0.000000}
\pgfsetfillcolor{diafillcolor}
\pgfsetfillopacity{1.000000}
\node[anchor=base west,inner sep=0pt,outer sep=0pt,color=dialinecolor] at (44.255000\du,21.669176\du){-1};
% setfont left to latex
\definecolor{dialinecolor}{rgb}{0.000000, 0.000000, 0.000000}
\pgfsetstrokecolor{dialinecolor}
\pgfsetstrokeopacity{1.000000}
\definecolor{diafillcolor}{rgb}{0.000000, 0.000000, 0.000000}
\pgfsetfillcolor{diafillcolor}
\pgfsetfillopacity{1.000000}
\node[anchor=base west,inner sep=0pt,outer sep=0pt,color=dialinecolor] at (44.255000\du,23.680010\du){~1};
% setfont left to latex
\definecolor{dialinecolor}{rgb}{0.000000, 0.000000, 0.000000}
\pgfsetstrokecolor{dialinecolor}
\pgfsetstrokeopacity{1.000000}
\definecolor{diafillcolor}{rgb}{0.000000, 0.000000, 0.000000}
\pgfsetfillcolor{diafillcolor}
\pgfsetfillopacity{1.000000}
\node[anchor=base west,inner sep=0pt,outer sep=0pt,color=dialinecolor] at (15.800000\du,9.500010\du){~-1~~~1~~~~~~~-1~~~1~~-1~~-1};
% setfont left to latex
\definecolor{dialinecolor}{rgb}{0.000000, 0.000000, 0.000000}
\pgfsetstrokecolor{dialinecolor}
\pgfsetstrokeopacity{1.000000}
\definecolor{diafillcolor}{rgb}{0.000000, 0.000000, 0.000000}
\pgfsetfillcolor{diafillcolor}
\pgfsetfillopacity{1.000000}
\node[anchor=base west,inner sep=0pt,outer sep=0pt,color=dialinecolor] at (15.755000\du,39.427510\du){~-1~~~1~~~~~~~-1~~~1~~-1~~-1};
% setfont left to latex
\definecolor{dialinecolor}{rgb}{0.000000, 0.000000, 0.000000}
\pgfsetstrokecolor{dialinecolor}
\pgfsetstrokeopacity{1.000000}
\definecolor{diafillcolor}{rgb}{0.000000, 0.000000, 0.000000}
\pgfsetfillcolor{diafillcolor}
\pgfsetfillopacity{1.000000}
\node[anchor=base west,inner sep=0pt,outer sep=0pt,color=dialinecolor] at (20.400000\du,39.400010\du){1};
\pgfsetlinewidth{0.100000\du}
\pgfsetdash{}{0pt}
\pgfsetbuttcap
{
\definecolor{diafillcolor}{rgb}{0.000000, 0.000000, 1.000000}
\pgfsetfillcolor{diafillcolor}
\pgfsetfillopacity{1.000000}
% was here!!!
\pgfsetarrowsend{stealth}
\definecolor{dialinecolor}{rgb}{0.000000, 0.000000, 1.000000}
\pgfsetstrokecolor{dialinecolor}
\pgfsetstrokeopacity{1.000000}
\pgfpathmoveto{\pgfpoint{38.400033\du}{40.009760\du}}
\pgfpatharc{367}{352.3}{120.625390\du and 120.625390\du}
\pgfusepath{stroke}
}
\pgfsetlinewidth{0.100000\du}
\pgfsetdash{}{0pt}
\pgfsetbuttcap
{
\definecolor{diafillcolor}{rgb}{0.000000, 0.000000, 1.000000}
\pgfsetfillcolor{diafillcolor}
\pgfsetfillopacity{1.000000}
% was here!!!
\pgfsetarrowsend{stealth}
\definecolor{dialinecolor}{rgb}{0.000000, 0.000000, 1.000000}
\pgfsetstrokecolor{dialinecolor}
\pgfsetstrokeopacity{1.000000}
\pgfpathmoveto{\pgfpoint{15.397711\du}{32.073879\du}}
\pgfpatharc{117}{65}{35.724437\du and 35.724437\du}
\pgfusepath{stroke}
}
}
\end{tikzpicture}
\end{Large}
\caption{Graphical representation of the CPU layout. The grids are via periodic boundary conditions vertically and horizontally in rings connected. Between the Monte-Carl steps each core writes its inner bond (blue) in the outer bond (red) of its neighbour.}
\label{ranks}
\end{figure}

\newpage
\section{Evaluation}
We evaluated the mean runtimes of 10 simulations with 10000 steps each. They were plotted for different number of processes against the length $s=N_x=N_y=\sqrt{V}$ in log-log scale (Abb. \ref{plot}).

\begin{figure}[h]
\scalebox{1.1}{%
\input{ising.pgf}
}
\caption{Runtime against length of the grid for three different numbers of processes and corressponding fitting functions in log-log scale.}
\label{plot}
\end{figure}

The algorith only works if the number of cores is a quadratic number.

With $s$ the volume rises quadratically, so does the number of spins to be updated. Therefore, the algorithm has the complexity $\mathcal{O}(s^2)$.

The fitting constant in the exponent does not hold as expected $2$. That is because we made the assumption in the fitting procedure, that parts of the algorithm, which rise linearly with $s$, are negliable. That is not true, especially for small grids. 

If we set the exponent to a fixed value and fit only the prefactor, we have ppeedups of 3.5 and 4.84. The deviation from the maximum speedup of 6 in the second case comes from the fact, that the simulation has to be started with a quadratic number of processes (that means 9). Half of the time half of the cores remain inactive ($6 - 1.5 = 4.5$).

\end{onehalfspace}
\end{document}